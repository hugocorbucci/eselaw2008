\documentclass[12pt]{article}

\usepackage{sbc-template}
\usepackage[latin1]{inputenc}
\usepackage{indentfirst}
\usepackage{lscape}
\usepackage{latexsym}
\usepackage{amssymb}
\usepackage{textcomp}
\usepackage{graphics,url}
\usepackage[T1]{fontenc}
\usepackage{hyperref}

\sloppy

\title{Open Source and Agile:\\Two worlds that should have a closer
  interaction}

\author{Hugo Corbucci\inst{1} and Alfredo Goldman\inst{1}}

\address{Instituto de Matem�tica e Estat�stica (IME)\\Universidade de S�o
  Paulo (USP) - Brazil
\email{\{corbucci,gold\}@ime.usp.br}
}

\begin{document}

\maketitle

\section{Introduction}

Agile methods and open source software communities share similar
cultures with different approaches to overcome problems. Although
several professionals are involved in both worlds, neither agile
methodologies are as strong as they could be in open source
communities nor those communities provide strong contributions to
agile methods. This work intends to identify and expose the obstacles
that separate those communities in order to extract the best of them
and improve both sides with suggestions of tools and development
processes.

Typical community open source projects (to be defined on Section
\ref{subsec:os-def}) usually receive the collaboration of many
geographically distant people \cite{Dempsey1999} and are organized
around a leader and she is the only hierarquical structure in the
group. At first, this argument could indicate that such projects are
not candidates for the use of agile methods since some basic values
seem to be damaged. For example, the distance and diversity separating
developers surely deteriorate communication, one of the most important
values within agile methods. However, most open source projects share
principles with the Agile Manifesto \cite{AgileManifesto}. Being ready
for changes, working with continuous feedback, delivering real
features, respecting collaborators and users and facing challenges are
expected attitudes from agile developers naturally found in the Free
and Open Source Software (FOSS) communities.

During a workshop \cite{OOPSLA07} about ``No Silver Bullets''
\cite{Brooks1987} held at OOPSLA 2007, Agile methods and Open Source
Software Development were mentioned as two failed silver bullets
having both brought great benefit to the software community. During
the same workshop the question was raised whether the use of several
failed silver bullets simultaneously could not, in fact, raise
production levels in an order of magnitude. This paper is an attempt
to suggest one of those mergings to partially tackle software
development problems.

The topics discussed in this work comprehend only a subset of projects
that are said agile or open source. Section \ref{sec:definitions}
presents definitions used to understand this work for both of those
terms.  Section \ref{sec:foss} will present some aspects of major open
source communities that could be improved with agile practices and
principles. The next section (Section \ref{sec:agile}) will focus on
the problems agile methods pose when dealing with distributed teams
and scaling to big teams which have somehow already been addressed in
open source development. Finally, Section \ref{sec:conclusion} will
present our current and future work.

\section{Definition}
\label{sec:definitions}

In order to start talking about open source and agile methods, it is
necessary to first define what is understood by such words in the
following work. Agile methods are defined at Section
\ref{subsec:agile-def} while the most controversial definition, the
open source one, is given at Section \ref{subsec:os-def}.

\subsection{Agile methods definition}
\label{subsec:agile-def}

This work will consider that any software engineering method that
follows the principles in the Agile Manifesto \cite{AgileManifesto} is
an agile method. Focus will be layed on the most known methods such as
Extreme Programming \cite{XP01,XP02}, Scrum
\cite{Schwaber2001,Schwaber2004} and the Crystal family
\cite{Cockburn2002}. Closely related ideas will also be mentioned from
the wider Lean philosophy \cite{Ohno1998} and its application to
software development \cite{Poppendieck2005,Poppendieck2003}.

\subsection{Open source definition}
\label{subsec:os-def}

The terms ``Open source software'' and ``Free software'' will be
considered the same in this work although they have important
diferences in their specific context \cite[Ch. 1, Free Versus Open
source]{Fogel2005}. Projects will be considered to be open source (or
free) if their source code is available and modifiable by anyone with
the required technical knowledge without prior consentment from the
original author and without any charge. Note that this definition is
closer from the free software idea than the open source one.

Another restriction will be that projects started and controlled by a
company do not fit in this definition of open source. The reason
beside this is that projects controlled by companies, whether they
have a public source code and accept external collaboration or not,
can be run with any software engineering methodology since the company
can enforce it to her employees. Some methodologies will work better
to attract external contributions but the company is still in control
of its own team and can maintain the software without external
collaboration.

Considering this definition, it is important to characterize the
people involved in such projects. In 2002, the FLOSS Project
\cite{FlossProject} published a report about a survey they conducted
regarding free or open source software contributors. Their collected
data \cite{FlossStats} shows on question 42 that 78.77\% of the
contributors are employed or self-employed and that only 50.82\% of
the open source community are software developers while 24.76\% do not
earn their main income with software development (question 10). This
is relevant because it bases the result that 78.78\% of open source
collaborators think working on those projects is more joyful (question
22.2) and 42.3\% find it also better organized (question 22.4).

Those data suggest that imposing a more traditional software
development process on such community has not been succesful. Another
survey \cite{Reis2003} points out that 74\% of open source projects
have teams up to 5 people and 62\% work with each other over the
Internet and never met physically or personally. It is therefore
critical for those projects to have an adequate software process that
fits those characteristics and is not a burden on the volunteer work.

\section{Is Open source Agile?}
\label{sec:foss}

Open source communities could almost be considered agile and they
indeed were by Martin Fowler in his first version of ``The New
Methodology'' \cite{Fowler00orig}. The methods that Eric Raymond
describes in ``The Cathedral and the Bazaar'' \cite{Raymond1998} lack a
more precise definition but several ideas could be related to the
Agile Manifesto. The next four subsections will discuss the relations
of open source to the four principles of the manifesto and the fifth
one will summarize points where open source could improve towards
agility.

\subsection{Individuals and interactions over processes and tools}
\label{subsec:first-princ}

Several researches regarding open source software development present
a reasonable amount of tools used by the developers to maintain
communication between their members. Reis \cite{Reis2003} shows that
version control software, the website and mailing lists are the most
used tools (over 65\% of the project use them) to communicate with the
users and in the team.

Several of the tools used in the open software process are also used
in the agile software development, such as version control programs
and websites. \textbf{The processes and tools} are, however, just a
means to achieve a goal: ensuring a stable and welcoming environment
to create software collaboratively.

Although open source businesses are growing stronger, the very essence
of the community around the software is to have \textbf{individuals
  that interact} in order to produce what interests them
\cite{Reis2003}. In those communities, the interaction is usually
related to source code collaboration and documentation elaboration
regardless of the business model. Those activities are responsible for
driving the whole process and modifying the tools to better fit their
needs.

\subsection{Working software over comprehensive documentation}
\label{subsec:second-princ}

According to Reis \cite{Reis2003}, 55\% of open source projects update
and revise their documentation frequently and 30\% maintain documents
that explain how parts of it work or how is it organized. Those
results show that documentation is not considered to be essential on
the projects.

More recently, Oram \cite{Oram2007} presented the results of a survey
conducted by O'Reilly showing that free documentation is increasingly
being written by volunteers. It means that \textbf{comprehensive
  documentation} grows with the community around the \textbf{working
  software}, as users encounter problems to complete a specific
action. According to Oram's work, the most important reasons for
contributors to write documentation is to improve the community or
their own personal growth. This motives explain why open source
documentation usually comprehend the most common problems and explain
how to use the most frequently used features but are somehow faulty to
provide details about less popular features.

\subsection{Customer collaboration over contract negotiation}
\label{subsec:third-princ}

\textbf{Contract negotiation} is still only a problem to very few open
source projects since a huge number of them do not involve
contracts. On the other hand, those involving contracts are usually
based on a service concept in which the customer hires a programmer or
company to develop a certain feature for a small amount of
time. Although this business model does not ensure that the customer
will collaborate, it may shorten time between conversations, therefore
improving feedback and reducing the strength of long and rigid
contracts.

The key point here is that collaboration is the basis of open source
projects. The customer is involved as much as he desires to
be. \textbf{Customers can collaborate} but they are not especially
encouraged or forced to do so. This might be related to the small
amount of experience this communities has with customer
relationships. However, several successful projects rely on fast
answers to the demanded features from users.  In this case customer
collaboration allied with responsiveness are specially powerful.

\subsection{Responding to change over following a plan}
\label{subsec:fourth-princ}

Open source projects tend to have a plan of milestones or releases
but, in most cases, those plans are always short term plans. Even when
long term plans exist, they are not the main guidelines followed by
the developer team. They are only goals sought without any pressure to
be met.

Being too demanding about \textbf{following a plan} can drag a whole
project down in the open source world. The main reason is the highly
competitive environment of this universe where only the best projects
survive. The \textbf{ability of each project to adapt and respond to
  changes} is crucial to determine those who survive. No marketing
campaign or business deal can save a project from abandonment if it
cannot compete with a newcomer that adapts more quickly to user needs.

\subsection{What is missing on open source?}
\label{subsec:os-summary}

Although several points of the Agile Manifesto are followed within
open source communities, nothing is certain because there is no such
thing as an open source method. Raymond's description \cite{Raymond1999}
is a great example of how the process can work but it does not
discriminate guidelines and practices to be followed. If a full open
source agile method description was written with the use of compatible
tools merging the ideas presented by Raymond, it would follow the same
selection rules as the projects. If successful, its adoption would
then spread around the community improving and correcting it over
time.

Communities created around FOSS projects involve users, developers,
and sometimes even clients working together to craft the best software
possible. The absence of such community around a program usually
denounces a recent project or one that is dying. Those signs mean that
the development team must be very attentive to its software community
which shows the health of the project. Nowadays, concerns related
to this aspect of FOSS development are not specifically considered by
the most known agile methods.

\section{Agile going Open source}
\label{sec:agile}

At Agile 2008, Mary Poppendieck led a workshop with Christian Reis to
discuss successful practices in an open source project that could not
be found in Agile methods. The goal was to capture some essential
principles that were applied to open source projects and could help
agile methods. A short summary of the discussion can be found in
section \ref{subsec:foss-over-agile}.  Thinking the other way around,
agile methodologies lack some special solutions related to open source
development. Finally section \ref{subsec:agile-self-improv} will
shortly present some benefits that agile would receive from attempts
to solve those problems.

\subsection{FOSS principles agile should learn from}
\label{subsec:foss-over-agile}

The workshop, intitled ``Open Source Meets Agile - What can each teach
the other?'', was coordinated by Mary Poppendieck and had Christian
Reis as the invited guest. Reis is a Brazilian open source developer
working at Canonical Inc. on the development of LaunchPad, the project
management software for Ubuntu Linux distribution. The workshop
started with Reis' presentation on how LaunchPad is developed. A few
points were highlighted as important differences and will be presented
on the following list.
\begin{itemize}
\item The commiter role

  Part of the value that was identified in open source was the role of
  commiter.  A commiter is a person that have rights to add source
  code to the trunk branch of the version control repository. The
  trunk branch is the portion of the code that is packaged to form a
  new version of that software. It means that the software community
  trusts the commiter to evaluate source code. This is open source's
  way to have most parts of the software source code reviewed to
  reduce the amount of errors and improve the code clarity.

  Most open source projects have a very small team of commiters.
  Frequently the project leader is the only commiter and all patches
  must be suggested to her. According to Riehle \cite{Riehle2007}, the
  commiter role is a valuable status recognition of quality
  contributions and is an explicit promotion while going from user to
  contributor is an implicit promotion since it is only the acceptance
  of a patch from the commiter.

  Agile methods entrust this role to every developer and it was
  suggested in the workshop that it might be good to have some sort of
  control to the trunk branch to ensure simplicity of the production
  source code. In most agile methods, a team should have a leader (a
  Scrum Master in Scrum, a Coach in Extreme Programming, etc...) that
  is more experienced in some aspect than the rest of team. This
  leader should have technical knowledge to discuss issues with the
  developers and remind them of the practices they should follow.

  It looks like a natural suggestion that the team's leader assumed
  the role of commiter. It would allow for an external review of the
  generated source code ensuring a higher level of clarity. This could
  support the pair programming code review not by reducing the amount
  of errors but by ensuring a cleaner code.

\item Public results

  Another important point was the publicity of all results regarding
  the project. According to Reis, non open source software can also
  benefit from public bug tracking and test results although they will
  have to accept some level of code detail to be exposed. Having such
  public tools encourages users to participate in the development
  process since they understand how the development is improved.

  In agile software development, bug tracking and test results are
  important information for the development team but no methodology
  clearly state that the client or user should be directly in contact
  with those tools. However, most say that the client should be
  considered part of the development team which can mean he should use
  those tools as the rest of the team does. The most used tools are
  very crude when considered from a non-developer perspective since
  few of them attribute a business meaning to their results. A few
  initiatives regarding tests exist on tools \cite{RSpec,JBehave}
  related to Behaviour Driven Development \cite{North2006} to produce
  better reports and bug tracking systems have been improving over
  time.

  But publicity is not restrained to bugs or tests. Discussions
  between members of the project and even with outsiders are always
  logged in the mailing lists archives. Personal discussions are
  strongly disencouraged to favor external comments and ideas. Those
  logs help building a documentation for future users as well as
  creating a quick feedback system to newcomers. The practice also
  serves as a tool to improve respect between parts since all decision
  are archived and saved for future access.

  % Criticize Agile methods here. They have co-location and do not
  % log ANYTHING. All is lost once the whiteboard is erased.

\item Cross reviewing

  An interesting practice that LaunchPad's team use is the cross
  revision. They do not pair program due to the geographic separation
  between the team's members but they demand that every API
  (Application Programming Interface) change is revised by a member of
  an outside team that uses LaunchPad, also known as, a developer
  user. From this practice, they benefit not only from the review but
  also from the diversity and they enforce documented communication
  (since all is done in mailing lists) and ensure a small explanation
  of the change's motivation.
\end{itemize}

\subsection{Agile contributions improving itself}
\label{subsec:agile-self-improv}

Most of the problems pointed out before are related to communication
issues triggered by the amount of people involved in the project and
their various knowledges and cultures. Although in open source those
matters are taken to a limit, distributed agile teams face some of the
same problems.

As the current situation of Internet makes evident, users are becoming
more and more important to the success or failure of a system. In such
perspective, providing feedback and absorbing suggestions and critics
will become essential to survival of a project. Just like the ability
to adapt placed agile methods on a very important position, the
ability to receive, select and incorporate suggestions from
communities will probably make the difference in the near
future. According to its own principles, agile methods should respond
to those changes and adapt to this growing matter. The best place to
start such work is within a community such as open source.

\section{Conclusion}
\label{sec:conclusion}

In this preliminary work we have shown several evidences that a
synergy with agile methods can improve software development on FOSS
projects. Several already adopt some agile techniques to be more
responsive to users but a complete description of a method that
considers all FOSS factors would surely increase adoption in those
communities. On the other hand, solving the problem is a challenge
that would consolidate agile methods to a distributed environment
relying on a large user community.

As part of this work, two surveys are planned. One to be conducted at
FISL (International Free Software Forum) 2009 to understand how much
open source developers and enthusiasts know about agile methods and
what keeps them from using them. The other one to be conducted at
Agile 2009 will try to discover how involved is the agile community
with open source development. Both surveys will be used to provide a
deeper understanding of the interaction between both communities and
how to improve it. Also, interviews with leaders of both communities
could help address more specific topics and gather suggestions and
support for the results of this work.

% Add a summary on helps

\bibliographystyle{plain} \bibliography{./biblio}

\end{document}
