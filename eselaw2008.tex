\documentclass[12pt]{article}

\usepackage{sbc-template}
\usepackage[latin1]{inputenc}
\usepackage{indentfirst}
\usepackage{lscape}
\usepackage{latexsym}
\usepackage{amssymb}
\usepackage{textcomp}
\usepackage{graphics,url}
\usepackage[T1]{fontenc}
\usepackage{hyperref}

\sloppy

\title{Open Source and Agile:\\Two worlds that should have a closer
  interaction}

\author{Hugo Corbucci\inst{1} and Alfredo Goldman\inst{1}}

\address{Instituto de Matem�tica e Estat�stica (IME)\\Universidade de S�o
  Paulo (USP) - Brazil
\email{\{corbucci,gold\}@ime.usp.br}
}

\begin{document}

\maketitle

\begin{abstract}
  Agile methods and open source software communities share similar
  cultures with different approaches to overcome problems. Although
  several people are involved in both worlds, neither agile
  methodologies are as strong as they could be in open source
  communities nor those communities are strong factors in agile
  methods. This work intends to identify and expose the obstacles that
  separate those communities in order to extract the best of them and
  improve both sides with suggestions of tools and development
  processes.
\end{abstract}

\section{Definitions}
\label{sec:definitions}

In order to start talking about open source and agile methods, it is
necessary to first define what is understood by such words in the
following work. Agile methods are more simply defined at section
\ref{subsec:agile-def} while the most controversial definition, the
open source one, is given at section \ref{subsec:os-def}.

\subsection{Agile methods definition}
\label{subsec:agile-def}

This work will consider that any software engineering methodology that
follows the principles in the Agile Manifesto \cite{AgileManifesto} is
an agile method. Focus will be layed on the most known methods such as
Extreme Programming \cite{XP01,XP02}, Scrum
\cite{Schwaber2001,Schwaber2004} and the Crystal family
\cite{Cockburn2002}. Closely related ideas will also be mentioned from
the wider Lean philosophy \cite{Ohno1998} and its application to
software development \cite{Poppendieck2005,Poppendieck2003}.

\subsection{Open source definition}
\label{subsec:os-def}

The terms ``Open source software'' and ``Free software'' will be
considered the same in this work although they have important
diferences in their specific context \cite[Ch. 1, Free Versus Open
source]{Fogel2005}. Projects will be considered to be open source (or
free) if their source code is available and modifiable by anyone with
the required technical knowledge without prior consentment from the
original author and without any charge (this definition is closer from
the free software idea than the open source one). The other
restriction is that only projects started from individual initiatives
will be considered to be run in an open source development fashion.

Projects controlled by companies, whether they have a public source
code and accept external collaboration or not, can be run with any
software engineering methodology since the company can enforce it to
her employees. Some methodologies will work better to attract external
contributions but the company is still in control of its own team.

With a community project, nobody can enforce volunteers to follow a
specific work way or those will abandon the project. In such
situation, the methodology must be accepted and embraced by all people
involved. Therefore the software engineering methods used must be
lightweight. %Achar uma referencia sobre lightweight vs heavyweight

\section{Introduction}
\label{sec:intro}

Open source projects usually receive the collaboration of many
geographically distant people who do not share any organizational
structure. At first, this argument could indicate that such projects
are not candidates for the use of agile methods since some basic
values seem to be damaged. For example, the distance and diversity
separating developers surely deteriorate communication, one of the
most important values within agile methods. However, most open source
projects share principles with the Agile Manifesto
\cite{AgileManifesto}. Being ready for changes, working with
continuous feedback, delivering real features, respecting
collaborators and users and facing challenges are expected attitudes
from agile developers naturally found in the Free and Open Source
Software (FOSS) communities.

During a workshop \cite{OOPSLA07} about ``No Silver Bullets''
\cite{Brooks1987} held at OOPSLA 2007, Agile methods and Open Source
Software Development were mentioned as two failed silver bullets
having both brought great benefit to the software community. During
the same workshop the question was raised whether the use of several
failed silver bullets simultaneously could not, in fact, raise
production levels in an order of magnitude. This paper is an attempt
to suggest one of those mergings to partially stop problems from
appearing unexpectedly.

Section \ref{sec:foss} will present some aspects of major open source
communities that could be improved with agile practices and
principles. The following section (Section \ref{sec:agile}) will focus
on the problems agile methods pose when dealing with distributed teams
and scaling to big teams which have somehow already been addressed in
open source development. Finally, Section \ref{sec:conclusion} will
present the work planned and being done.

\section{Is Open source Agile?}
\label{sec:foss}

Open source communities could almost be considered agile and they
indeed were by Martin Fowler in his first version of ``The New
Methodology'' \cite{Fowler00orig}. The methods that Eric Raymond
describes in ``The Cathedral and the Bazaar'' \cite{Raymond98} lack a
more precise definition but several ideas could be related to the
Agile Manifesto. The next four subsections will discuss the relations
of open source to the four points principles of the manifesto and the
fifth one will summarize points where open source could improve
towards agility.

\subsection{Individuals and interactions over processes and tools}
\label{subsec:first-princ}

Project processes usually include feature freezing, version branches,
commit reviews and several other good practices or rules. Most of the
time, tools are used to enable those practices and other ones are
present and widely used. Several of the tools used in the open
software process are also used in the agile software development, such
as version control programs. \textbf{The processes and tools} are,
however, just a means to achieve a goal: ensuring a stable and
welcoming environment to create software collaboratively.

Although open source businesses are growing stronger, the very essence
of the community around the software is to have \textbf{individuals
  that interact} in order to produce what interests them. In those
communities, the interaction is usually related to source code
collaboration and documentation elaboration regardless of the business
model. Those activities are responsible for driving the whole process
and modifying the tools to better fit their needs.

\subsection{Working software over comprehensive documentation}
\label{subsec:second-princ}

A lot of open source projects are heavily criticized for their
documentations or the lack of it. This comes from the fact that most
developers are not committed on writing documentation. More likely,
they prefer to have a neat software that is intuitive for users. The
result is that new projects hardly have any sort of documentation
except the minimum required for the own developer team to be able to
work.

\textbf{Comprehensive documentation} grows with the community that
builds around the \textbf{working software}, as users encounter
problems to complete a specific action. It is frequent to have
documentation written by volunteer users to help their
colleagues. This work generates documents in a language that users
understand but that only deal with common problems. Specific problems
and solutions are much harder to find.

\subsection{Customer collaboration over contract negotiation}
\label{subsec:third-princ}

\textbf{Contract negotiation} is still only a problem to very few open
source projects since a huge number of them do not involve
contracts. On the other hand, those involving contracts are usually
based on a service concept in which the customer hires a programmer or
company to develop a certain feature for a small amount of
time. Although this business model does not ensure that the customer
will collaborate, it may shorten time between conversations, therefore
improving feedback and reducing the strength of long and rigid
contracts.

The key point here is that collaboration is the basis of open source
projects. The customer is involved as much as he desires to
be. \textbf{Customers can collaborate} but they are not especially
encouraged or forced to do so. This might be related to the small
amount of experience this communities has with customer
relationships. However, several successful projects rely on fast
answers to demanded features from users.  In this case customer
collaboration allied with responsiveness are specially powerful.

\subsection{Responding to change over following a plan}
\label{subsec:fourth-princ}

Open source projects tend to have a plan of milestones or releases
but, in most cases, those plans are always short term plans. Even when
long term plans exist, they are not the main guidelines followed by
the developer team. They are only goals sought without any pressure to
be met.

Being too demanding about \textbf{following a plan} can drag a whole
project down in the open source world. The main reason is the highly
competitive environment of this universe where only the best projects
survive. The \textbf{ability of each project to adapt and respond to
  changes} is crucial to determine those who survive. No marketing
campaign or business deal can save a project from abandonment if it
cannot compete with a newcomer that adapts more quickly to user needs.

\subsection{What is missing on open source?}
\label{subsec:os-summary}

Although several points of the Agile Manifesto are followed within
open source communities, nothing is certain because there is no such
thing as an open source method. Raymond's description is a great
example of how the process can work but it does not discriminate
guidelines and practices to be followed. If a full open source agile
method description was written with the use of compatible tools
merging the ideas presented by Raymond, it would follow the same
selection rules as the projects. If successful, its adoption would
then spread around the community improving and correcting it over
time.

Communities created around FOSS projects involve users, developers,
and sometimes even clients working together to craft the best software
possible. The absence of such community around a program usually
denounces a recent project or one that is dying. This means that the
development team must be very attentive to this community since it
shows how well the project is going. Nowadays, concerns related to
this aspect of FOSS development are not specifically considered by the
most known agile methods.

\section{Agile going Open source}
\label{sec:agile}

At Agile 2008, Mary Poppendieck led a workshop with Christian Reis to
discuss successful practices in an open source project that could not
be found in Agile methods. The goal was to capture some essential
principles that were applied to open source projects and could help
agile methods. A short summary of the discussion can be found in
section \ref{subsec:foss-over-agile}.  Thinking the other way around,
agile methodologies lack some special solutions related to open source
development. Section \ref{subsec:agile-lacks} discuss with more
details some specific points of FOSS development that would need a
specific approach from an agile perspective. Finally section
\ref{subsec:agile-self-improv} will shortly present some benefits that
agile would receive from attempts to solve those problems.

\subsection{FOSS principles agile should learn from}
\label{subsec:foss-over-agile}

The workshop, intitled ``Open Source Meets Agile - What can each teach
the other?'', was coordinated by Mary Poppendieck and had Christian
Reis as the invited guest. Reis is a Brazilian open source
developer working at Canonical Inc. on the development of LaunchPad,
the project management software for Ubuntu Linux distribution. The
workshop started with Reis' presentation on how LaunchPad is
developed. During the presentation and once it was done, the group had
some questions that helped understand the practices that the team
follows.

Part of the value that was identified in open source was the role of
Commiter. That person, in an open source development process, is the
one responsible for maintaining the quality of the trunk branch of the
version control repository. According to Riehle %/cite
, the Commiter role is a valuable status recognition of quality
contributions. Agile methods entrust this role to every developer and
it was suggested in the workshop that it might be good to have some
sort of control to the trunk branch to ensure simplicity of the code.

Another important point was the publicity of all results regarding the
project. According to Reis, non open source software can also benefit
from public bug tracking and test reulsts although they will have to
accept some level of code detail to be exposed. Having such public
tools encourages users to participate in the development process since
they understand how things work. The same publicity exist regarding
the discussions between members of the project and even with
outsiders. This practice serves as a tool to improve respect between
parts since all decision are archived and saved for future access.

An interesting practice that LaunchPad's team use is the cross
revision. They do not pair program because of the geographic
separation between the team's members but they demand that every API
(Application Programming Interface) change is revised by a member of
an outside team that uses LaunchPad, also known as, a developer
user. From this practice, they benefit not only from the review but
also from the diversity and they enforce documented communication
(since all is done in mailing lists) and ensure a small explanation of
the change's motivation.

\subsection{Agile helping FOSS}
\label{subsec:agile-lacks}

Agile development so far has been described as a way to develop
software within companies with contracts and employees. Forming and
maintaining a community bounded to the system is the responsibility of
the marketing and sales people. As long as the contract exists, there
is no danger regarding the adoption of the project and its user
base. In a FOSS project, none of those factors are ensured at any
moment. Even if there is a contractor and there are employees, the
community must be kept active and welcoming. Addressing a community,
responding to its requirements and providing feedback to its members
is not an easy task. What, when and how to provide feedback must be
wisely chosen and is a time consuming activity that cannot be
undertaken.

\begin{itemize}
\item How to balance between customer requirement and community
  requirement?
\item How should providing feedback be handled within an iteration?
\item Should plans be made counting on external help or not?
\item If so, how to estimate expectations about external
  contributions?
\item What tools or measures should be used to make it easier for
  people to contribute?
\item How should commits be approved or denied?
\end{itemize}
     
Those are only a few questions that are unanswered when dealing with
open source communities using an agile method. This work intends to
provide a wider analysis of those issues to gather a more complete and
precise list of issues related to open source development that agile
methods do not provide answers for.

\subsection{Agile contributions improving itself}
\label{subsec:agile-self-improv}

Most of the problems pointed out before are related to communication
issues triggered by the amount of people involved in the project and
their various knowledges and cultures. Although in open source those
matters are taken to a limit, distributed agile teams face some of the
same problems. Evolving a software that will be used by many people
around the world with slightly different processes and laws may
require distributed agile teams working geographically distant with
specific local clients.

As the current situation of Internet makes evident, users are becoming
more and more important to the success or failure of a system. In such
perspective, providing feedback and absorbing suggestions and critics
will become essential to survival of a project. Just like the ability
to adapt placed agile methods to a very important position, the
ability to receive, select and incorporate suggestions from
communities will probably make the difference in the near
future. According to its own principles, agile methods should respond
to those changes and adapt to this growing matter. The best place to
start such work is within an extreme community such as open source.

\section{Conclusion}
\label{sec:conclusion}

In this preliminary work we have shown several evidences that a
synergy with agile methods can improve software development on FOSS
projects. Several already adopt some agile techniques to be more
responsive to users but a complete description of a method that
considers all FOSS factors would surely increase adoption in those
communities. On the other hand, solving the problem is a challenge
that would consolidate agile methods to a distributed environment
relying on a large user community.

As part of this work, two surveys are planned. One to be conducted at
FISL (International Free Software Forum) 2008 to understand how much
open source developers and enthusiasts know about agile methods and
what keeps them from using them. The other one to be conducted at
Agile 2008 will try to discover how involved is the agile community
with open source development. Both surveys will be used to provide a
deeper understanding of the interaction between both communities and
how to improve it. Also, interviews with leaders of both communities
could help address more specific topics and gather suggestions and
support for the results of this work.

\bibliographystyle{plain} \bibliography{./biblio}

\end{document}
