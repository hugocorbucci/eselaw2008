\documentclass[12pt]{article}

\usepackage{sbc-template}
\usepackage[latin1]{inputenc}
\usepackage{indentfirst}
\usepackage{lscape}
\usepackage{latexsym}
\usepackage{amssymb}
\usepackage{textcomp}
\usepackage{graphicx,url}
\usepackage[T1]{fontenc}
\usepackage{hyperref}

\sloppy

\title{Open Source and Agile:\\Two worlds that should have a closer
  interaction}

\author{Hugo Corbucci\inst{1} and Alfredo Goldman\inst{1}}

\address{Instituto de Matem�tica e Estat�stica (IME)\\Universidade de S�o
  Paulo (USP) - Brazil
\email{\{corbucci,gold\}@ime.usp.br}
}

\begin{document}

\maketitle

\begin{abstract}
  Agile methods and open source software communities share similar
  cultures with different approaches to overcome problems. Although
  several professionals are involved in both worlds, neither agile
  methodologies are as strong as they could be in open source
  communities nor those communities provide strong contributions to
  agile methods. This work identifies and exposes the obstacles that
  separate those communities in order to extract the best of them and
  improve both sides with suggestions of tools and development
  processes.
\end{abstract}

\section{Introduction}

Typical open source projects (to be defined on Section
\ref{subsec:os-def}) usually receive the collaboration of many
geographically distant people \cite{Dempsey1999} and are organized
around a leader which is the only hierarchical structure in the
group. At first glance, this argument could indicate that such
projects are not candidates for the use of agile methods since some
basic values seem to be missing. For example, the distance and
diversity separating developers surely deteriorate communication, one
of the most important values within agile methods. However, most open
source projects share some principles of the Agile Manifesto
\cite{AgileManifesto}. Being ready for changes, working with
continuous feedback, delivering real features, respecting
collaborators and users and facing challenges are expected attitudes
from agile developers naturally found in the Free and Open Source
Software (FOSS) communities.

During a workshop \cite{OOPSLA07} about ``No Silver Bullets''
\cite{Brooks1987} held at OOPSLA 2007, Agile methods and Open Source
Software Development were mentioned as two failed silver bullets
having both brought great benefit to the software community. During
the same workshop the question was raised whether the use of several
failed silver bullets simultaneously could not raise production levels
in an order of magnitude. This is an attempt to suggest one of those
merges to partially tackle software development problems.

The topics discussed in this work comprehend only a subset of projects
that are said agile or open source. Section \ref{sec:definitions}
presents the definitions used in our work.  Section \ref{sec:foss}
will present some aspects of major open source communities that could
be improved with agile practices and principles. The next section
(Section \ref{sec:agile}) will focus on the problems agile methods
pose when dealing with distributed teams and scaling to big teams
which have somehow already been addressed in open source
development. Finally, Section \ref{sec:conclusion} will summarize our
current work and present our future tasks.

\section{Definition}
\label{sec:definitions}

In order to start talking about open source and agile methods, it is
necessary to first define what is understood by such words. Agile
methods are defined at Section \ref{subsec:agile-def} while the open
source definition, more controversial, is given at Section
\ref{subsec:os-def}.

\subsection{Agile methods definition}
\label{subsec:agile-def}

This work will consider that any software engineering method that
follows the principles in the Agile Manifesto \cite{AgileManifesto} is
an agile method. Focus will be laid on the most known methods such as
Extreme Programming \cite{XP02}, Scrum \cite{Schwaber2004} and the
Crystal family \cite{Cockburn2002}. Closely related ideas will also be
mentioned from the wider Lean philosophy \cite{Ohno1998} and its
application to software development \cite{Poppendieck2005}.

\subsection{Open source definition}
\label{subsec:os-def}

The terms ``Open source software'' and ``Free software'' will be
considered the same in this work although they have important
differences in their specific context \cite[Ch. 1, Free Versus Open
source]{Fogel2005}. Projects will be considered to be open source (or
free) if their source code is available and modifiable by anyone with
the required technical knowledge without prior consent from the
original author and without any charge. Note that this definition is
closer from the free software idea than the open source one.

Another restriction will be that projects started and controlled by a
company do not fit in this definition of open source. The reason
beside this is that projects controlled by companies, whether they
have a public source code and accept external collaboration or not,
can be run with any software engineering methodology since the company
can enforce it to her employees. Some methodologies will work better
to attract external contributions but the company is still in control
of its own team and can maintain the software without external
collaboration.

Considering this definition, it is important to characterize the
people involved in such projects. In 2002, the FLOSS Project
\cite{FlossProject} published a report about a survey they conducted
regarding free or open source software contributors. Their collected
data \cite{FlossStats} shows on question 42 that 78.77\% of the
contributors are employed or self-employed and that only 50.82\% of
the open source community are software developers while 24.76\% do not
earn their main income with software development (question 10).  In
addition to those results, the survey presents the fact that 78.78\%
of the collaborators consider their open source tasks more joyful
(question 22.2) than their regular activities and 42.3\% also consider
them better organized (question 22.4). As a conclusion, we could say
that open source contributors perceive their activities both
pleasurable and effective.

One of the possibilities to have such sensations can be linked to the
liberty on the development, where there is no heavy process
attached. Another survey \cite{Reis2003} points out that 74\% of open
source projects have teams with up to 5 people and 62\% work with each
other over the Internet and never met physically or personally. It is
therefore critical for those projects to have an adequate software
process that fits those characteristics and is not a burden on the
volunteer work.

\section{Is Open source Agile?}
\label{sec:foss}

Open source communities could almost be considered agile and they
indeed were by Martin Fowler in his first version of ``The New
Methodology'' \cite{Fowler00orig}. The methods that Eric Raymond
describes in ``The Cathedral and the Bazaar'' \cite{Raymond1998} lack
a more precise definition but several ideas could be related to the
Agile Manifesto. The next four subsections will discuss the relations
of open source to the four principles of the manifesto and the fifth
one will summarize points where open source could improve towards
agility.

\subsection{Individuals and interactions over processes and tools}
\label{subsec:first-princ}

Several researches regarding open source software development present
a reasonable amount of tools used by the developers to maintain
communication between their members. Reis \cite{Reis2003} shows that
version control software, the website and mailing lists are the most
used tools (over 65\% of the projects use them) to communicate with
the users and in the team.

Several of the tools used in the open software process are also used
in the agile software development, such as version control programs
and websites. \textbf{The processes and tools} are, however, just a
mean to achieve a goal: ensuring a stable and welcoming environment to
create software collaboratively.

Although open source businesses are growing stronger, the very essence
of the community around the software is to have \textbf{individuals
  that interact} in order to produce what interests them
\cite{Reis2003}. In those communities, the interaction is usually
related to source code collaboration and documentation elaboration
regardless of the business model. Those activities are responsible for
driving the whole process and modifying the tools to better fit their
needs.

\subsection{Working software over comprehensive documentation}
\label{subsec:second-princ}

According to Reis \cite{Reis2003}, 55\% of the open source projects
update and revise their documentation frequently and 30\% maintain
documents that explain how parts of it work or how is it
organized. Those results show that documentation is considered
important but not the goal of the projects.

More recently, Oram \cite{Oram2007} presented the results of a survey
conducted by O'Reilly showing that free documentation is increasingly
being written by volunteers. It means that \textbf{comprehensive
  documentation} grows with the community around the \textbf{working
  software}, as users encounter problems to complete a specific
action. According to Oram's work, the most important reasons for
contributors to write documentation is to their personal growth or to
improve the community. It explains why open source documentation
usually comprehend the most common problems and explain how to use the
most frequently used features but are faulty to provide details about
less popular features.

\subsection{Customer collaboration over contract negotiation}
\label{subsec:third-princ}

\textbf{Contract negotiation} is still only a problem to very few open
source projects since a huge number of them do not involve
contracts. On the other hand, those involving contracts are usually
based on a service concept in which the customer hires a programmer or
company to develop a certain feature for a small amount of
time. Although this business model does not ensure that the customer
will collaborate, it may shorten the time between conversations,
therefore improving feedback and reducing the strength of long and
rigid contracts.

The key point here is that collaboration is the basis of open source
projects. The customer is involved as much as he desires to
be. \textbf{Customers can collaborate} but they are not especially
encouraged or forced to do so. This might be related to the small
amount of experience this communities has with customer
relationships. However, several successful projects rely on fast
answers to the demanded features from users.  In this case customer
collaboration allied with responsiveness are specially powerful.

\subsection{Responding to change over following a plan}
\label{subsec:fourth-princ}

Open source projects tend to have a plan of milestones or releases.
Several projects only count with short term plans and, even when long
term plans exist, they are not the main guidelines followed by the
developer team. They are only goals sought without any pressure to be
met. % TODO: Achar referencia

Being too demanding about \textbf{following a plan} can drag a whole
project down in the open source world. The main reason is the highly
competitive environment of this universe where only the best projects
survive. The \textbf{ability of each project to adapt and respond to
  changes} is crucial to determine those who survive. No marketing
campaign or business deal can save a project from abandonment if it
cannot compete with a newcomer that adapts more quickly to user needs.

\subsection{What is missing on open source?}
\label{subsec:os-summary}

Although several points of the Agile Manifesto are followed within
open source communities, nothing is certain because there is no such
thing as an open source method. Raymond's description
\cite{Raymond1999} is a great example of how the process can work but
it does not discriminate guidelines and practices to be followed. If a
full open source agile method description was written with the use of
compatible tools merging the ideas presented by Raymond, it would
follow the same selection rules as the projects. If successful, its
adoption would then spread around the community improving and
correcting it over time.

Communities created around FOSS projects involve users, developers,
and sometimes even clients working together to craft the best software
possible. The absence of such community around a program usually
denounces a recent project or one that is dying. Those signs mean that
the development team must be very attentive to its software community
which shows the health of the project. Nowadays, concerns related to
this aspect of FOSS development are not specifically considered by the
most known agile methods.

\section{Agile going Open source}
\label{sec:agile}

At Agile 2008, Mary Poppendieck led a
workshop\footnote{http://submissions.agile2008.org/node/376} with
Christian Reis entitled ``Open Source Meets Agile - What can each
teach the other?''. Its goal was to discuss successful practices in an
open source project that could not be found in Agile methods. The goal
was to capture some essential principles that were applied to open
source projects and could improve agile methods. A short summary of
the discussion can be found in section \ref{subsec:foss-over-agile}.
Thinking the other way around, agile methodologies lack some special
solutions related to open source development. Finally section
\ref{subsec:agile-improve-os} will shortly present some benefits could
be introduced in agile methods.

\subsection{FOSS principles agile should learn from}
\label{subsec:foss-over-agile}

Reis is a Brazilian open source developer working at Canonical Inc. on
the development of LaunchPad, the project management software for
Ubuntu Linux distribution. The workshop started with Reis'
presentation on how LaunchPad is developed. Three main points were
highlighted during the discussions that followed the presentation and
will be describe in the next subsection. The first one (Section
\ref{subsubsec:commiter}) describes and discusses the role of
commiter. The second one (Section \ref{subsubsec:publicity}) presents
the benefits of having a transparent and public process and the last
(Section \ref{subsubsec:crossrev}) talks about cross reviewing systems
used to ensure communication and clarity of the code.
\subsubsection{The commiter role}
\label{subsubsec:commiter}

Part of the value that was identified in open source was the role of
commiter.  A commiter is a person that have rights to add source code
to the trunk branch of the version control repository. The trunk
branch is the portion of the code that is packaged to form a new
version of that software. It means that the software community trusts
the commiter to evaluate source code. This is open source's way to
have most parts of the software source code reviewed to reduce the
amount of errors and improve the code clarity.

Most open source projects have a very small team of commiters.
Frequently the project leader is the only commiter and all patches
must be suggested to her. According to Riehle \cite{Riehle2007}, there
are three stages in open source common hierarchy. The first level is
to be a user. Being a user, you get the right to use the software and
report bugs and request features. The second level is being a
contributor. The promotion between the first and the second level is
implicit. It happens when a commiter accepts your patch and send it to
the trunk branch. Usually, nobody except the commiter and the
contributor know about this change. The third role is the commiter
one. At this level, the transition is explicit. Contributors and
commiters vouch for a contributor and recognize publicly the overall
quality of his work. Reaching the commiter level is a very valuable
promotion that means you produce good quality code and is involved in
the project's development.

Agile methods entrust this role to every developer and it was
suggested in the workshop that it might be good to have some sort of
control to the trunk branch to ensure simplicity of the production
source code. In most agile methods, a team should have a leader (a
Scrum Master in Scrum, a Coach in Extreme Programming, etc...) that is
more experienced in some aspect than the rest of team. This leader
should have technical knowledge to discuss issues with the developers
and remind them of the practices they should follow.

It looks like a natural suggestion that the team's leader may assume
the role of commiter. It would allow for an external review of the
generated source code ensuring a higher level of clarity. This could
support the pair programming code review not by reducing the amount of
errors but by ensuring a cleaner code. On the other hand, the team's
leader could become the bottleneck for code production or would have
to abandon his other tasks to fulfill this one. An idea here would be
to have a small set of developers being commiters and this role would
circulate.
\subsubsection{Public results}
\label{subsubsec:publicity}

Another important point was the publicity of all results regarding the
project. According to Reis, non open source software can also benefit
from public bug tracking and test results although they will have to
accept some level of code detail to be exposed. Having such public
tools encourages users to participate in the development process since
they understand how the development is improved.

In agile software development, bug tracking and test results are
important information for the development team but no methodology
clearly state that the client or user should be directly in contact
with those tools. However, most say that the client should be
considered part of the development team which can mean he should use
those tools as the rest of the team does. The most used tools are very
crude when considered from a non-developer perspective since few of
them attribute a business meaning to their results. Few initiatives
regarding tests exist on tools \cite{RSpec,JBehave} related to
Behaviour Driven Development \cite{North2006} to produce better
reports and bug tracking systems have been improving over time.

But publicity is not restrained to bugs or tests. Discussions between
members of the project and even with outsiders are always logged in
the mailing lists archives. Personal discussions are strongly
discouraged to favor external comments and ideas. Those logs help
building a documentation for future users as well as creating a quick
feedback system to newcomers. The practice also serves as a tool to
improve respect between parts since all decision are archived and
saved for future access.

This sort of traceability is one of the weak points of agile
methods. Most of them suggest that design evolves with time according
to the needs and that this evolutions flows naturally on whiteboards
or flip charts. The problem with this approach is that whiteboards are
erased and flip charts are recycled. Even when those are persisted
somehow (by pictures, transcription or even in the code itself), the
discussion that led to the solution is lost. Talking is a very
effective way to communicate but is also very ephemeral. Once the
conversation is over, it is hard to quickly reach the precise
information you are searching for. Emails have a much lower
communication bandwidth but gain on their ability to search. In a short
term, it is evident that talking is more effective than writing when
handling small teams but in a middle or long period, the gains might
outcome (as they do in open source) the losses.
\subsubsection{Cross reviewing}
\label{subsubsec:crossrev}

The third point that Reis presented was pretty specific to
LaunchPad. Since LaunchPad is a platform used by other teams to
develop their own project, when there is an API (Application
Programming Interface) modification, a member from an external team
that uses the software (preferably a different one each time) is asked
to review the API change and its motivation. Such change cannot be
added to the trunk branch of the repository unless it has been
approved by the external reviewer. They call this a cross review of API
changes or, simply, cross review.

This practice tackle a few problems at once. The commiter role
partially solves the code review problem that is addressed with pair
programming on agile methods. Having a cross review ensures that the
API will be approved by two different developers.

It also improves greatly documentation regarding that API since the
conversation between the project developer and the user is logged
through the mailing list. This way, future or other users can read and
understand why the API was changed and how to use it when it is best
suited for them. It also helps future changes and simplifications
since it is easy to check whether the condition at the time of the
change still holds on new versions.

Finally, it also helps involving the user or client in the
architectural decisions as well as ensures that he agrees on the
changes. This helps discovering possible requirement problems and
correcting them before they get implemented in the main code
base. Obviously such practice can only apply to some level when the
user is not a developer. Having an external review will help ensuring
API clarity and document the changes but it might not detect
requirement problems if the reviewer is not a user or client.
\subsection{Agile contributions to improve Open Source}
\label{subsec:agile-improve-os}

Most of the problems pointed out before are related to communication
issues triggered by the amount of people involved in the project and
their various knowledge and cultures. Although in open source those
matters are taken to a limit, distributed agile teams face some of the
same problems \cite{Sutherland2007,Maurer2002}.

As Beck suggests \cite{Beck2008}, tools can improve the adoption and
use of agile practices and, therefore, improve a development
process. A fair amount of work has been directed to distributed pair
programming tools \cite{Xpairtise} and studies \cite{Nagappan2003} but
very few tools have been produced to support other practices. Since
communication is at stack, a few other practices are related to it in
agile methods. The following subsections will present those practices
and the tools to improve the open source experience with agile
development.
\subsubsection{Informative Workspace}
\label{subsubsec:inform-worksp}

This practice suggests that an agile team should work in an
environment that gives them information regarding their work. Beck
assigns a specific role, the tracker role, to the entity that should
maintain this information available and updated to the team. With
co-located teams, the tracker usually collects metrics \cite{Sato2007}
automatically and selects a few of them to present in the
workspace. Most of the objective metrics are related to the code base
while subjective ones depend on developers' opinions.

Collecting those data is not a hard task but they are usually time
consuming and do not produce immediate benefit to the software
advance. This is probably the reason why it is very rare to find an
open source project with updated metrics and data in their website. A
tool that could improve such scenario would be web-based plug in system
with a built-in metrics collection as well as a way to add and present
new metrics. Such tool should be available into open source forge
applications to allow projects to easily connect them to their
repository and web page.
\subsubsection{Stories}
\label{subsubsec:stories}

Regarding the planning system, extreme programming suggest that
requirements should be collected in user story cards. The goal is to
minimize the amount of effort required to discover the next step to
make and being able to easily change those requirements priority over
time. Open source projects usually are based on bug tracking systems
to store those requirements. A feature missing is reported as a bug
that should be corrected and discussions and patch suggestions are
submitted related to that ``bug''. The problem with this approach is
that changing priority and setting a release plan is very time
consuming and relies on non documented assumptions (such as ``this
release should solve priority bugs over 8''). It is also very hard to
picture an overall view of all the requirements.

Discovering what are the main priorities for the team quickly and
being able to change those priority according to the community's
feedback is key to develop a working software. To help achieve this, a
tool should be implemented to allow bugs to be seen as movable
elements on a release planning. It should also have the bug's priority
and content set by the community in a similar way as Wikipedia manages
its articles.
\subsubsection{Retrospective}
\label{subsubsec:retrospect}

This practice suggests that the team should meet in the same working
space periodically to discuss the way the project is going. There are
two issues in such practice in open source software teams. The first
one is to have all members of the team present at the same time. The
second one is to have them interact collectively in a shared area
placing notes on time stating problems and good things they felt.

When the team is co-located, this is usually done a meeting room with
a huge time line and coloured post-its. Our suggestion is to develop a
web-based tool to allow such interaction relating the time line to the
code repository base. The team would be able to annotate
asynchronously the time line. The team leader would occasionally
generate a report sent to all members as well as posted in the
informative workspace.
\subsubsection{Stand up meetings}
\label{subsubsec:stand-up}

This practice, originally suggested in the Scrum methodology, shares
the same problem as the retrospective. It involves having the team
together at the same time. Several open source projects already have a
partial solution to this practice using an IRC channel to centralize
the discussions during development time. Although it does not ensure
everyone gets to know what other members are doing, it helps
synchronizing work.

To ensure members get the required knowledge, we suggest that those
IRC channels should be logged and present the last few messages to
newcomers at every log in. It should also be possible for members to
leave notes from that channel to the bug tracking system as well as
messages to other contributors. On IRC channel, this sort of solution
is usually performed by a bot which we intend to associate to the
project website.

\section{Conclusion}
\label{sec:conclusion}

In this preliminary work we have shown several evidences that a
synergy with agile methods can improve software development on FOSS
projects. Several already adopt some agile techniques to be more
responsive to users but a complete description of a method that
considers all FOSS factors would surely increase adoption in those
communities. On the other hand, solving the problem is a challenge
that would consolidate agile methods to a distributed environment
relying on a large user community.

As part of this work, two surveys are planned. One to be conducted at
FISL (International Free Software Forum) 2009 to understand how much
open source developers and enthusiasts know about agile methods and
what keeps them from using them. The other one to be conducted at
Agile 2009 will try to discover how involved is the agile community
with open source development. Both surveys will be used to provide a
deeper understanding of the interaction between both communities and
how to improve it.

\section{Acknowledgements}

This work was supported by the QualiPSo project \cite{Qualipso}. We
would like to thank Christian Reis for his help, interesting
discussions and support as well as Mariana V. Bravo and Danilo Sato
for reviewing this work.

\bibliographystyle{plain} \bibliography{./biblio}

\end{document}
